\documentclass[11pt]{article}

\usepackage{amsmath, amssymb}
\usepackage{graphicx}
\usepackage{hyperref}
\usepackage{geometry}
\geometry{margin=1in}

\title{Constraint Field Theory}
\author{William Shacklett}
\date{\today}

\begin{document}
\maketitle

\begin{abstract}
Constraint Field Theory (CFT) explores whether many observed limits in complex systems
can be modeled as a structured, saturating constraint field.
This paper introduces definitions, axioms, minimal formalism,
testable predictions, and falsification criteria,
supported by toy simulations.
\end{abstract}

\section{Introduction}

Across many domains, complex systems exhibit persistent limits.
Physical systems display bounded responses to increasing energy input,
biological systems saturate under resource pressure,
cognitive systems encounter attention and processing ceilings,
and engineered software systems degrade or destabilize beyond certain loads.

These limits are often treated as domain-specific artifacts:
material strength limits in physics,
carrying capacities in biology,
or performance bottlenecks in software.
While such explanations are locally effective,
they provide little insight into why saturation, stability,
and boundary behavior appear so consistently across disparate systems.

A recurring pattern is that increasing input does not lead to proportional output.
Instead, systems enter regimes of diminishing returns,
plateaus, or abrupt transitions.
Understanding the structure of these regimes is critical for prediction,
control, and long-horizon reliability.

This paper explores whether these recurring behaviors can be modeled
using a common abstraction:
a constraint field that encodes how limits arise,
propagate, and saturate within a system.

\section{Motivation and Scope}

The motivation for Constraint Field Theory (CFT) is not to replace
domain-specific models, but to complement them with a unifying perspective
on limits and stability.

Rather than focusing on microscopic mechanisms,
CFT operates at the level of reachable states.
It asks not how a system moves,
but which states remain accessible as constraints accumulate.

CFT is intentionally minimal.
It does not posit a new physical force,
a universal conserved quantity,
or a privileged substrate.
Instead, it proposes that many systems can be usefully described
by a structured field of constraints whose effects are observable
through saturation, coupling, and boundary behavior.

The scope of CFT is therefore limited.
It aims to:
\begin{itemize}
\item characterize saturation and diminishing returns,
\item explain correlated constraint effects,
\item and provide stable observables for long-horizon assessment.
\end{itemize}

CFT does not attempt to:
\begin{itemize}
\item derive fundamental constants,
\item replace detailed dynamical models,
\item or assert metaphysical interpretations of constraint.
\end{itemize}

If successful, CFT should reduce explanatory complexity
by revealing shared structure across systems
that are otherwise treated as unrelated.
If unsuccessful, its failure modes should be explicit and testable.

\section{Definitions}

This section defines the core terms used throughout Constraint Field Theory (CFT).
Terminology is fixed prior to formal development to avoid ambiguity.

\subsection{Constraint}
A constraint is any rule, boundary, or limitation that restricts the set of
reachable states of a system.

Constraints may arise from physical laws, resource limitations,
structural dependencies, or design choices.
They are treated as real features of systems rather than artifacts
of incomplete knowledge.

\subsection{Constraint Field}
A constraint field is a structured distribution of constraints
defined over space, time, or abstract state space.

The constraint field encodes how limitations vary across a system
and how they influence reachable states locally and globally.

Notation placeholders include:
\begin{itemize}
\item $C(x,t)$ for a constraint field over spacetime,
\item $C(s)$ for a constraint field over abstract state space.
\end{itemize}

No assumption is made about the microscopic origin of the field.

\subsection{Saturation}
Saturation refers to the phenomenon in which increasing input,
drive, or effort produces diminishing or null changes
in system behavior due to constraint limits.

Saturation manifests as plateaus, diminishing returns,
or stable boundary regimes.

\subsection{Coupling}
Coupling describes dependencies between constraints,
such that modifying one constraint alters the effective influence
of others.

Coupling may be asymmetric, nonlinear, or delayed,
and need not correspond to direct physical interaction.

\subsection{Observable}
An observable is a measurable scalar or low-dimensional quantity
derived from the constraint field.

Observables summarize constraint behavior in a stable manner
and enable comparison across time, conditions, or systems.

\section{Axioms}
% Mirrors theory/02_axioms.md

\section{Minimal Formalism}
% Mirrors theory/03_minimal_math.md

\section{Predictions}
% Mirrors theory/04_predictions.md

\section{Falsification Criteria}
% Mirrors theory/05_falsification.md

\section{Toy Models and Examples}
% Pulls from examples/

\section{Discussion}
% Implications, limits, and future work

\section{Conclusion}

\bibliographystyle{plain}
\bibliography{references}

\end{document}
