\documentclass[11pt]{article}

\usepackage{amsmath, amssymb}
\usepackage{graphicx}
\usepackage{hyperref}
\usepackage{geometry}
\geometry{margin=1in}

\title{Constraint Field Theory}
\author{William Shacklett}
\date{\today}

\begin{document}
\maketitle

\begin{abstract}
Constraint Field Theory (CFT) explores whether many observed limits in complex systems
can be modeled as a structured, saturating constraint field.
This paper introduces definitions, axioms, minimal formalism,
testable predictions, and falsification criteria,
supported by toy simulations.
\end{abstract}

\section{Introduction}
% Why limits, saturation, and stability appear across domains

\section{Motivation and Scope}
% What CFT explains and explicitly does not explain

\section{Definitions}
% Mirrors theory/01_definitions.md

\section{Axioms}
% Mirrors theory/02_axioms.md

\section{Minimal Formalism}
% Mirrors theory/03_minimal_math.md

\section{Predictions}
% Mirrors theory/04_predictions.md

\section{Falsification Criteria}
% Mirrors theory/05_falsification.md

\section{Toy Models and Examples}
% Pulls from examples/

\section{Discussion}
% Implications, limits, and future work

\section{Conclusion}

\bibliographystyle{plain}
\bibliography{references}

\end{document}
